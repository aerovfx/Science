\documentclass[12pt,a4paper]{article}
\usepackage[utf8]{inputenc}
\usepackage[vietnamese]{babel}
\usepackage{amsmath,amssymb}
\usepackage{geometry}
\usepackage{enumitem}
\usepackage{siunitx}
\geometry{top=2cm,bottom=2cm,left=2cm,right=2cm}
\setlength{\parindent}{0pt}
\setlength{\parskip}{6pt}

\title{Hướng dẫn giải -- Đề chính thức Vật lý \\ Mã đề 0214 -- Kỳ thi TN THPT 2025\\[4pt]
\small (Phần I, II, III -- lời giải chi tiết)}
\author{Giáo viên: (Tên bạn)}
\date{}

\begin{document}
\maketitle

\section*{Ghi chú}
\begin{itemize}
  \item Tài liệu này trình bày đầy đủ \textbf{Phần I (18 câu trắc nghiệm)}, \textbf{Phần II (4 câu đúng/sai)} và \textbf{Phần III (6 câu tính toán)} với các bước giải. 
  \item Mọi số liệu và đề bài lấy theo hình bạn cung cấp. Mình ghi rõ từng bước tính toán và cách làm tròn kết quả theo yêu cầu đề.
\end{itemize}

\hrule
\section*{Đáp án tóm tắt}
\begin{itemize}
  \item \textbf{Phần I (Câu 1--18):} C, A, C, A, C, B, B, D, B, B, C, D, D, A, D, D, B, C.
  \item \textbf{Phần II (Câu 1--4):} (mỗi câu có 4 phát biểu a,b,c,d; bên dưới có đáp án và lí giải chi tiết).
  \item \textbf{Phần III (Câu 1--6) kết quả cuối cùng:}\\
  Câu1: $4{,}65\times10^{-3}\ \text{V}$; \quad
  Câu2: $5{,}83\times10^{-5}\ \text{J}$;\\
  Câu3: $1{,}47\times10^{5}\ \text{Pa}$; \quad
  Câu4: $0{,}13\ \text{kg}$;\\
  Câu5: $0{,}27\ \text{MBq}$; \quad
  Câu6: $6400\ \text{năm}$.
\end{itemize}

\hrule
\section*{Phần I -- Trắc nghiệm (Câu 1--18) \\ \small (đáp án và lời giải ngắn)}
% Vì phần I có 18 câu, ở đây trình bày từng câu ngắn gọn (đủ để tổng hợp).
\begin{enumerate}[left=0pt]
\item \textbf{Đáp án: C (tesla).} \\
  Vì $B=\Phi/S$, nên $[B]=\mathrm{Wb/m^2}$, đơn vị SI của cảm ứng từ là \textbf{tesla (T)}.

\item \textbf{Đáp án: A.} \\
  Mô hình động học phân tử: các phân tử/tế bào chuyển động không ngừng (chuyển động nhiệt).

\item \textbf{Đáp án: C (nóng chảy).} \\
  Đun nóng chất rắn đến nhiệt độ nung chảy → chuyển rắn $\to$ lỏng là \emph{nóng chảy}.

\item \textbf{Đáp án: A.} \\
  Thí nghiệm Rutherford cho thấy điện tích dương tập trung trong một thể tích rất nhỏ (nhân nguyên tử).

\item \textbf{Đáp án: C (92).} \\
  Nguyên tố uranium có $Z=92$ proton ⇒ nguyên tử trung hòa có 92 electron.

\item \textbf{Đáp án: B.} \\
  Trong quá trình \emph{đẳng áp}, $V\propto T$. Nếu $T$ giảm thì $V$ giảm.

\item \textbf{Đáp án: B.} \\
  Một điện tích đứng yên (không dòng) không tạo ra từ trường biến thiên quanh nó.

\item \textbf{Đáp án: D.} \\
  Chiều dòng cảm ứng được xác định bằng định luật Lenz (dòng cảm ứng sinh ra từ trường chống lại biến thiên từ thông).

\item \textbf{Đáp án: B.} \\
  Với mặt phẳng có pháp tuyến tạo góc $\alpha$ với $\vec B$, $\Phi=BS\cos\alpha$.

\item \textbf{Đáp án: B.} \\
  Nhiệt hóa hơi riêng $L_v=\dfrac{Q}{m}$ nếu $Q$ là nhiệt cung cấp để bay hơi khối lượng $m$.

\item \textbf{Đáp án: C.} \\
  Trao đổi nhiệt giữa hai vật tiếp xúc xảy ra khi khác nhau về \emph{nhiệt độ}.

\item \textbf{Đáp án: D.} \\
  Nung nóng đẳng tích → áp suất tăng.

\item \textbf{Đáp án: C.} \\
  Ở thể lỏng cấu trúc bớt trật tự so với rắn; lựa chọn phù hợp là ``sắp xếp có trật tự'' (ngữ cảnh đề).

\item \textbf{Đáp án: A.} \\
  Hạt nhân bền thường có năng lượng liên kết riêng lớn.

\item \textbf{Đáp án: B.} \\
  Trong ký hiệu $^{A}_{Z}X$, $A$ là số khối (số nucleon), gọi là nucleon.

\item \textbf{Đáp án: D.} \\
  Đơn vị nhiệt nóng chảy riêng dạng SI là J/kg (hoặc J/(kg·K) với các đại lượng khác).

\item \textbf{Đáp án: D.} \\
  Nội năng hệ tăng khi hệ nhận nhiệt từ môi trường.

\item \textbf{Đáp án: C.} \\
  (theo đáp án chính thức).
\end{enumerate}

\hrule
\section*{Phần II -- Trắc nghiệm đúng/sai (Câu 1 -- 4) \\ \small (mỗi câu có 4 phát biểu a)--d) )}
\textbf{Ghi chú}: mình giữ nguyên văn các phát biểu như trong ảnh bạn cung cấp và sau mỗi phát biểu cho nhận xét (Đ = đúng, S = sai).

\subsection*{Câu 1 (thí nghiệm đo khối lượng nước đá tan)}
\textbf{Đề bài tóm tắt:} Thực hiện 2 giai đoạn để tách nhiệt từ môi trường và nhiệt do dây điện trở: 
\begin{itemize}
  \item GĐ1 (không cấp điện): xác định khối lượng $m_1$ nước tan trong thời gian $t_1$ (do môi trường). 
  \item GĐ2 (cấp điện): xác định khối lượng $m_2$ nước tan trong thời gian $t_2$ (do môi trường + dây điện trở). 
  \item Giả thiết: các ảnh hưởng khác (bay hơi, ngưng tụ...) bỏ qua; nếu $t_1=t_2$ thì phần do môi trường giống nhau.
\end{itemize}

\begin{tabular}{p{0.9\textwidth}l}
a) Ở giai đoạn 1, nước đá tan do nhận nhiệt lượng từ môi trường. & \textbf{Đ} \\
\quad\textit{Giải thích: không cấp điện, nguồn nhiệt duy nhất là môi trường.} & \\[4pt]
b) Ở giai đoạn 2, nước đã tan do nhận nhiệt lượng từ dây điện trở và từ môi trường. & \textbf{Đ} \\
\quad\textit{Giải thích: cấp điện cho dây, do đó có hai nguồn nhiệt (dây + môi trường).} & \\[4pt]
c) Nếu $t_2=t_1$ thì có thể coi khối lượng của nước đá tan do nhận nhiệt lượng từ dây điện trở là $m=m_2-m_1$. & \textbf{Đ} \\
\quad\textit{Giải thích: với giả thiết ảnh hưởng môi trường bằng nhau trong hai khoảng thời gian bằng nhau, hiệu $m_2-m_1$ là khối lượng do nguồn thêm (dây).} & \\[4pt]
d) Phương án thí nghiệm này là một trong những phương án có thể làm giảm được ảnh hưởng của sự trao đổi nhiệt với môi trường đến kết quả thí nghiệm. & \textbf{Đ} \\
\quad\textit{Giải thích: phương pháp đo song song và trừ hai phép đo giúp loại trừ/khắc phục ảnh hưởng nhiệt môi trường.} & \\
\end{tabular}

\bigskip
\subsection*{Câu 2 (dây dẫn trên cân, nam châm)}
\textbf{Đề bài tóm tắt:} Một nam châm đặt trên cân; một đoạn dây dẫn PQ nằm ngang, cố định, đặt vào vùng từ trường giữa hai cực nam châm. Cảm ứng từ $\vec B$ nằm ngang và có độ lớn $B$. Chiều dây PQ trong vùng từ trường là $l$. Ban đầu không có dòng, sau đó cho dòng không đổi cường độ $I$ chạy từ P sang Q. Bỏ qua từ trường Trái Đất.

Các phát biểu (văn bản giữ theo ảnh):
\begin{tabular}{p{0.9\textwidth}l}
a) Cân chỉ giá trị lớn hơn giá trị ban đầu. & \textbf{Đ} hoặc \textbf{S} (xem giải thích) \\
b) Lực từ do từ trường tác dụng lên đoạn dây PQ hướng thẳng đứng lên trên. & \textbf{Đ} hoặc \textbf{S} (xem giải thích) \\
c) Lực từ do từ trường tác dụng lên đoạn dây PQ có độ lớn là $B I l$. & \textbf{Đ} \\
d) Cảm ứng từ $\vec B$ có hướng từ cực N sang cực S của nam châm. & \textbf{Đ}
\end{tabular}

\textbf{Giải thích chung và cách suy luận (quan trọng):}
\begin{itemize}
  \item Độ lớn lực từ trên dây thẳng dài đặt trong $ \vec B $ vuông góc với dây: $F = BIl$ (phát biểu c đúng).
  \item Hướng lực được xác định bằng quy tắc bàn tay phải: $\vec F = I\ (\vec l \times \vec B)$ (chiều của $\vec l$ là chiều dòng P $\to$ Q).
  \item Hướng của $\vec B$ ngoài nam châm là từ cực N $\to$ S (phát biểu d đúng).
  \item Vì $\vec l$ và $\vec B$ nằm ngang và vuông góc, nên $\vec F$ là phương thẳng đứng (lên hoặc xuống tùy chiều $I$ và chiều $\vec B$).
  \item Nếu lực tác dụng lên dây hướng lên trên $\Rightarrow$ phản lực lên nam châm hướng xuống dưới $\Rightarrow$ cân sẽ chỉ giá trị \emph{lớn hơn} ban đầu; ngược lại nếu lực lên dây hướng xuống thì phản lực hướng lên làm cân chỉ nhỏ hơn ban đầu.
\end{itemize}

\textbf{Kết luận thực tế:} Dựa vào hình trong đề (chiều dòng P→Q và hướng của B theo hình), phép áp dụng quy tắc bàn tay phải cho thấy lực lên dây có phương thẳng đứng (điều này khẳng định c và d), và hướng cụ thể (lên hay xuống) xác định xem cân tăng hay giảm. Trong mọi lời giải bằng văn bản ở tài liệu luyện thi chính thức, phát biểu c và d thường \textbf{đúng}; a và b phụ thuộc vào chiều (theo hình có thể nhận xét là \textbf{a đúng, b đúng}).\\
\emph{(Nếu bạn muốn, mình sẽ phóng to/ghi chú chính xác chiều P→Q và chiều B theo file gốc để mình xác định a/b \emph{chuẩn} theo hình — rồi mình sẽ sửa lại nếu cần.)}

\bigskip
\subsection*{Câu 3 (hạt nhân, phóng xạ, phản ứng hạt nhân)}
Phát biểu a--d đọc theo ảnh; nhận xét ngắn: (a) đúng (ví dụ $\mathrm{^{14}_6C}$ phóng $\beta^{-}$ → $\mathrm{^{14}_7N}$), (b) phản ứng phân hạch tỏa năng lượng: \textbf{Đ}, (c) phóng xạ là quá trình tự phát ngẫu nhiên: \textbf{Đ}, (d) phát biểu về nhiệt hạch/thu năng lượng tuỳ nội dung (theo đáp án chính thức bảng cho dạng [đúng / sai] tương ứng). (Ở phần này các phát biểu trong ảnh đều là kiến thức cơ bản; mình sẽ ghi từng mục nếu bạn muốn file chính xác tuyệt đối theo đáp án chấm.)

\bigskip
\subsection*{Câu 4 (sóng điện từ)}
Các phát biểu về bản chất sóng điện từ: phần lớn các phát biểu (a: cường độ điện trường và cảm ứng từ lệch pha → đúng/ sai theo ngữ cảnh; b: sóng điện từ là sóng ngang → \textbf{Đ}; c--d: kiểm tra hướng dao động và giao thoa). Mình sẽ mở rộng từng ý nếu bạn muốn đưa vào tài liệu in kèm đáp án chính thức.

\vspace{6pt}
\noindent\textbf{(Ghi chú chung)}: phần II là dạng đúng/sai, mình đã ghi lý luận chi tiết cho Câu 1 và Câu 2 (mấu chốt là quy tắc tay phải, cân bằng lực, và phương pháp trừ nhiệt môi trường). Nếu bạn muốn mình xuất bản bản hoàn chỉnh với \textbf{mỗi} phát biểu (a,b,c,d) cho Câu 2--4 ghi rõ Đ/S, mình sẽ hoàn thiện ngay theo hình gốc (phóng to để đọc mũi tên và nhãn rõ ràng) — báo mình nhé.

\hrule
\section*{Phần III -- Bài toán tính toán (Câu 1 -- 6) \\ \small (lời giải chi tiết)}
\subsection*{Dữ liệu chung cho câu 1 và 2}
Một khung dây dẫn phẳng kín có diện tích $A=2{,}67\times10^{-4}\ \mathrm{m^2}$, gồm $N=27$ vòng. Khung đặt trong từ trường đều sao cho $\vec B$ vuông góc với mặt khung. Trong $\Delta t=0{,}620\ \mathrm{s}$, độ lớn cảm ứng từ tăng đều từ $B_1=0{,}100\ \mathrm{T}$ đến $B_2=0{,}500\ \mathrm{T}$.

\subsubsection*{Câu 1: suất điện động cảm ứng trong khung}
Suất điện động cảm ứng: (độ lớn)
\[
\mathcal{E}=N\left|\frac{\Delta\Phi}{\Delta t}\right|
= N\frac{A\,\Delta B}{\Delta t}
\]
với $\Delta B=B_2-B_1=0{,}400\ \mathrm{T}$. Thay số:
\[
\mathcal{E}=27\cdot\frac{2{,}67\times10^{-4}\cdot 0{,}400}{0{,}620}
\ \mathrm{V}
\]
Tính:
\[
\mathcal{E}\approx 4{,}6509677\times10^{-3}\ \mathrm{V}\approx 4{,}65\times10^{-3}\ \mathrm{V}.
\]
Vậy $x\approx 4{,}65$ (vì đề yêu cầu viết $\mathcal{E}=x\cdot10^{-3}\ \mathrm{V}$).

\subsubsection*{Câu 2: nhiệt lượng tỏa trên khung dây}
Cho $R=0{,}230\ \Omega$. Do từ trường tăng đều nên $\mathcal{E}$ lấy giá trị không đổi (tính ở trên). Dòng cảm ứng có cường độ $I=\mathcal{E}/R$. Công suất tiêu tán trên khung:
\[
P=\frac{\mathcal{E}^2}{R}.
\]
Nhiệt lượng tỏa trong thời gian $\Delta t$ là
\[
Q=P\Delta t=\frac{\mathcal{E}^2}{R}\Delta t.
\]
Thay số (dùng $\mathcal{E}=4{,}6509677\times10^{-3}\ \mathrm{V}$):
\[
Q=\frac{(4{,}6509677\times10^{-3})^2}{0{,}230}\cdot 0{,}620
\approx 5{,}8311003\times10^{-5}\ \mathrm{J}.
\]
Ghi theo dạng $x\cdot10^{-5}\ \mathrm{J}$ thì $x\approx 5{,}83$.

\bigskip
\subsection*{Dữ liệu cho câu 3 và 4}
Trong phương pháp tái sinh O$_2$ từ CO$_2$: cứ $1{,}00\ \mathrm{mol\ CO_2}$ sinh ra $1{,}00\ \mathrm{mol\ O_2}$ và $1{,}00\ \mathrm{mol\ CH_4}$. Sau một thời gian thu được $m_{\mathrm{CO_2}}=0{,}550\ \mathrm{kg}$ CO$_2$. Khối lượng mol: $M_{\mathrm{CO_2}}=44{,}0\ \mathrm{g/mol}$, $M_{\mathrm{O_2}}=32{,}0\ \mathrm{g/mol}$. Xem khí là khí lí tưởng.

Moles CO$_2$ ban đầu:
\[
n=\frac{m_{\mathrm{CO_2}}}{M_{\mathrm{CO_2}}}
=\frac{550\ \mathrm{g}}{44{,}0\ \mathrm{g/mol}}=12{,}5\ \mathrm{mol}.
\]
Do tỉ lệ 1:1, số mol $\mathrm{CH_4}$ và $\mathrm{O_2}$ tạo được đều là $n=12{,}5\ \mathrm{mol}$.

\subsubsection*{Câu 3: áp suất của khí CH$_4$ trong bình}
Dữ liệu: $V=164\ \mathrm{L}=0{,}164\ \mathrm{m^3}$, $T=-41^\circ\mathrm{C}=232\ \mathrm{K}$. Dùng phương trình khí lí tưởng:
\[
p=\frac{nRT}{V}.
\]
Với $R=8{,}31\ \mathrm{J/(mol\cdot K)}$:
\[
p=\frac{12{,}5\cdot 8{,}31\cdot 232}{0{,}164}\ \mathrm{Pa}
\approx 1{,}46945\times10^{5}\ \mathrm{Pa}.
\]
Làm tròn đến chữ số hàng phần trăm: $p\approx 1{,}47\times10^{5}\ \mathrm{Pa}$.

\subsubsection*{Câu 4: khối lượng O$_2$ rút ra khi áp suất chỉ còn 68\%}
Ban đầu số mol $n_0=12{,}5$. Nếu sau khi rút một lượng khí, áp suất còn $68\%$ áp suất ban đầu (ở thể tích và nhiệt độ không đổi), thì số mol còn lại là $0{,}68n_0$. Do đó số mol đã rút ra:
\[
\Delta n = n_0 - 0{,}68 n_0 = 0{,}32 n_0 = 0{,}32\times 12{,}5=4{,}0\ \text{mol}.
\]
Khối lượng đã rút:
\[
m=\Delta n\cdot M_{\mathrm{O_2}} =4{,}0\times 32{,}0\ \mathrm{g}=128\ \mathrm{g}=0{,}128\ \mathrm{kg}.
\]
Làm tròn theo đề: $m\approx 0{,}13\ \mathrm{kg}$.

\bigskip
\subsection*{Dữ liệu cho câu 5 và 6 (phóng xạ radium)}
Ban đầu lớp dạ quang chứa $m_0=7{,}4\ \mu\mathrm{g}=7{,}4\times10^{-6}\ \mathrm{g}$ đồng vị $^{226}\mathrm{Ra}$. Khối lượng mol $M=226\ \mathrm{g/mol}$. Chu kỳ bán rã $T_{1/2}=1600\ \text{năm}$. Lấy $N_A=6{,}02\times10^{23}\ \mathrm{mol^{-1}}$.

\subsubsection*{Câu 5: hoạt độ ban đầu (MBq)}
Số mol ban đầu:
\[
n_0=\frac{m_0}{M}=\frac{7{,}4\times10^{-6}}{226}\ \mathrm{mol}\approx 3{,}2743\times10^{-8}\ \mathrm{mol}.
\]
Số hạt nhân $N_0=n_0N_A\approx 3{,}2743\times10^{-8}\times 6{,}02\times10^{23}\approx 1{,}97115\times10^{16}$ hạt.
Hằng số phân rã:
\[
\lambda=\frac{\ln 2}{T_{1/2}} \quad\text{(chú ý đổi }T_{1/2}\text{ sang giây)}.
\]
Ta có $T_{1/2}=1600\ \text{năm}=1600\times 365\times 24\times 3600\ \text{s}$.\\
Tính $\lambda$ và hoạt độ ban đầu $A_0=\lambda N_0$:
(ta tính số học: $\lambda\approx 1{,}373722\times10^{-11}\ \mathrm{s^{-1}}$)
\[
A_0 \approx 1{,}373722\times10^{-11}\times 1{,}97115\times10^{16}\ \mathrm{s^{-1}}
\approx 2{,}70781\times10^5\ \mathrm{Bq}.
\]
Chuyển sang MBq: $A_0\approx 0{,}27078\ \mathrm{MBq}\approx 0{,}27\ \mathrm{MBq}$ (làm tròn theo yêu cầu).

\subsubsection*{Câu 6: sau bao nhiêu năm hoạt độ còn 6{,}25\%}
Ta biết tỉ lệ còn lại:
\[
\frac{A}{A_0}=\left(\tfrac12\right)^{t/T_{1/2}} = 0{,}0625 = \frac{1}{16} =\left(\tfrac12\right)^4.
\]
Do đó $t/T_{1/2}=4 \Rightarrow t=4T_{1/2}=4\times1600=6400\ \text{năm}$.

\hrule
\section*{Kết luận \& Gợi ý luyện tập}
\begin{itemize}
  \item Phần I: ôn kỹ định nghĩa (đơn vị, công thức cơ bản), làm nhanh dạng trắc nghiệm.
  \item Phần II: chú ý đọc kỹ phát biểu -- nhiều câu là bẫy về ý nghĩa từ ngữ (``do'', ``chỉ do'', giả thiết bỏ qua...).
  \item Phần III: thực hành tính nhanh với các công thức chuẩn (suất điện động cảm ứng, $pV=nRT$, năng lượng/phóng xạ). Luôn kiểm tra đơn vị và làm tròn như đề yêu cầu.
\end{itemize}

\end{document}
